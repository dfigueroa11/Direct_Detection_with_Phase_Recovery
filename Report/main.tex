\documentclass[letterpaper, 11pt]{book}

%---------------------Paquetes a utilizar---------------------------%
\usepackage[left=3.6cm,right=2.54cm,top=2.54cm,bottom=2.54cm]{geometry}%Formato general de la página y márgenes 
\usepackage[utf8]{inputenc}                 %Encoding
\usepackage[english]{babel}                 %Español
\usepackage{csquotes}                       %Herramientas para citar
\usepackage[style = ieee]{biblatex}          %Manejo de bibliografía
\usepackage{dsfont}
\usepackage{bm}
\usepackage{amsmath}                        %Algunos entornos matemáticos (como align)
\usepackage{amssymb}                        %Algunos entornos matemáticos (como align)
\usepackage[labelfont=bf]{caption}          %Formato de figuras
\usepackage{subcaption}                     %subfigures
\usepackage{graphicx}                       %Manejo de imágenes
\usepackage{hyperref}                       %Uso de hipervínculos
\usepackage{titlesec}                       %Para modificar el formato de los títulos
\usepackage{fancyhdr}                       %Encabezados y pie de páginas personalizados
\usepackage{lipsum}
\usepackage{nomencl}
\makenomenclature
\renewcommand{\nomname}{List of symbols and abbreviations}
\usepackage{siunitx}
\usepackage{slashbox}
\DeclareMathOperator*{\argmax}{arg\,max}
%---------Configurar algunas propiedades del documento-------------%
%\renewcommand{\baselinestretch}{1.5}        %Interliniado de 1.5   

%\addto\captionsspanish{\renewcommand{\listfigurename}{Lista de figuras}}
%\addto\captionsspanish{\renewcommand{\listtablename}{Lista de tablas}}
%\addto\captionsspanish{\renewcommand{\figurename}{Figura}}
%\addto\captionsspanish{\renewcommand{\tablename}{Tabla}}

%\renewcommand\thefigure{\thechapter-\arabic{figure}}
% \renewcommand\thetable{\thechapter-\arabic{table}}

\def\imagesize{0.8\textwidth}

\renewcommand{\chaptermark}[1]{\markboth{#1}{}}%Definición de cómo se almacena el nombre del capítulo


%-----------Formato de títulos de capítulo--------------------------%
\titleformat{\chapter}[display]
{\normalfont\huge\bfseries}{\chaptertitlename\ \thechapter}{20pt}{\Huge}                                     %Propiedades por defecto de los títulos de capítulo

\titlespacing*{\chapter}{0pt}{100pt}{24pt}  %Dar espaciado de 100pt en la parte superior y 24pt en la parte inferior del título de capítulo

%-------------Formato de pies de página y encabezados---------------%

\fancypagestyle{plain}{
    \fancyhf{}
    \renewcommand{\headrulewidth}{0pt} 
    \renewcommand{\footrulewidth}{0pt}}     %Estilo de la primera página de cada capítulo 



\pagestyle{fancy}
\fancyhf{}
\fancyhead[RE,LO]{\thepage}
\fancyhead[RO,LE]{\leftmark}  


%-----------------Archivo con referencias bibliográficas------------%
\addbibresource{references.bib} % Biblatex


%------------------Inicio del documento-----------------------------%
\begin{document}
\frontmatter

\begin{titlepage}
    \begin{center}
          
      	\begin{figure}[ht]
      		\includegraphics[width=0.3\textwidth]{images/EscudoUNAL.png}
      		\centering
      	\end{figure}
            
        \vspace{3.5cm}
            
        \LARGE{\textbf{Título del proyecto}}
        
        
        \vspace{3.5cm}
        \large{\textbf{Nombre de xlx estudiante}}
        \vfill
        
        
        \normalsize{Universidad Nacional de Colombia\\
					Facultad de , Departamento \\
					Ciudad, Colombia\\
					Año}
            
    \end{center}
\end{titlepage}
\let\cleardoublepage\clearpage
\begin{titlepage}
    \begin{center}            
        \LARGE{\textbf{Título del proyecto}}  
            
        \vspace{3.5cm}
        \large{\textbf{Nombre de xlx estudiante}}
            
        \vspace{3.5cm}
        \normalsize{Trabajo presentado como requisito parcial para optar al título de:\\
        \textbf{Título esperado}}    
     
        \vspace{2cm}
        \normalsize{Directorx:\\ Nombre de quien dirige}
        
        \vspace{1.2cm}
        \normalsize{L\'inea de Investigaci\'on:\\Línea de investigación}
        \vfill
		\normalsize{Universidad Nacional de Colombia\\
					Facultad de , Departamento de \\
					Ciudad, Colombia\\
					Año}
            
    \end{center}
\end{titlepage}

\chapter*{Acknowledgments}
\chaptermark{Acknowledgments}





\chapter*{Abstract}
\addcontentsline{toc}{chapter}{Abstract} 
\chaptermark{Abstract}




\tableofcontents

\cleardoublepage
\phantomsection
\addcontentsline{toc}{chapter}{\listfigurename} 
\listoffigures

\cleardoublepage
\phantomsection
\addcontentsline{toc}{chapter}{\listtablename} 
\listoftables

\cleardoublepage
\phantomsection
\addcontentsline{toc}{chapter}{\nomname} 
\printnomenclature

\cleardoublepage


\mainmatter
\chapter{Introduction}
\chaptermark{Introduction}
\label{ch:introduction}


	

Square-law detection, also known as Direct detection (DD), is a detection scheme that measures the square magnitude of a complex wave form, clearly this scheme is a nonlinear waveform detection because of the square operation. Direct detection is widely used in different scientific fields, for example in crystallography, radio astronomy, biomedical spectroscopy, among others \cite{Tasbihi_Tukey}. In particular this scheme is often used in optical communications, specially in short-haul systems with length up to \SI{50}{\km} \cite{Agrawal_ch1} due to its simplicity, or even in shorter links up to \SI{10}{\km} for example in rack to rack communications in big data centers.\\
\nomenclature{DD}{Direct Detection}

In the recent years the interest of studying systems with DD is getting bigger again, because of the simplicity of the receivers, which are a promising low-cost alternative compared to the coherent detection systems \cite{Mecozzi_2018}. In this context two questions or problems arise. First, how good is a system with DD compared to a system with coherent detection in terms of the information capacity of the channels. And second, how to design a system that exploits the information capacity of the DD channel in the best possible way.\\

In this work we cover briefly two answers given to the first question, and then we review two systems proposed for a channel with DD. Then we propose a new decoder for the second system, with reduced complexity at the expenses of a slightly worse performance.




\section{Direct Detection}
\label{sec:Direct_Detection}

In optical communication DD is performed with a single photodiode, that convert the optical signal to an electric signal trough the photoelectric effect according to the next equation \cite{Agrawal_ch4}:
\begin{equation}
I_p = R_d\cdot P_{in}
\label{eq:photocurrent}
\end{equation}
where $I_p$ is the photocurrent, $P_{in}$ is the incident optical power (which is proportional to the square of the magnitud of the electric field, that is where the square-law term comes from), and $R_d$ is the so called responsivity of the photodetector, with units of \SI{}{\A/\W}.\\

The noise in the photodiode are generated primarily by two mechanisms, in the first place the shot noise, and in the second place thermal noise.\\

The shot noise is models the fact that the photocurrent consist of a stream of electrons generated at random times. Mathematically the current corresponding to the shot noise $i_s(t)$ is a stationary random process with Poisson distribution, but is usually approximated by a Gaussian distribution with variance given by \cite{Agrawal_ch4}:
\begin{equation}
\sigma_s^2 = 2qI_pB
\label{eq:shot_noise_varaince}
\end{equation}
where $q$ is the electron charge, $I_p$ the photocurrent, and $B$ the bandwidth of the system. $\sigma_s$ can be interpreted as the RMS value of the shot noise current $i_s(t)$.\\ 

The termal noise is generated by the movement  of electrons due to the ambient temperature, and the variance of the noise is given by \cite{Agrawal_ch4}:
\begin{equation}
\sigma_{th}^2 = \frac{4k_BT}{R_L} F_nB
\label{eq:thermal_noise_variance}
\end{equation}
where $k_B$ is the Boltzmann constant, $T$ is the temperature given in kelvin, $R_L$ is the load resistance, $B$ the bandwidth, and $F_n$ is the amplifier noise figure. \\

With this in mind the output current of the direct detection process is given by:
\begin{equation}
I(t) = I_p+i_s(t)+i_{th}(t)
\label{eq:DD_current}
\end{equation}
with $i_s(t)\sim\mathcal{N}(0,\sigma_s^2)$ and $i_{th}(t)\sim\mathcal{N}(0,\sigma_{th}^2)$
\nomenclature{$X\sim\mathcal{N}(\mu,\sigma^2)$}{$X$ has a Gaussian distribution with mean $\mu$ and variance $\sigma_2$}


\section{Capacity under direct detection}
\label{sec:capacity_under_direct_detection}

A communications channel that uses DD can retrieve only the information about the magnitude of the signal, in contrast a system with coherent detection can retrieve the magnitud and phase of the signal. This means that DD scheme ignores one of the two degrees of freedom, and hence it is reasonable to think that the capacity of this systems should be approximately half that of the system with coherent detection \cite{Mecozzi_2018, Tasbihi_Tukey, Tasbihi_Capacity}.\\

However in \cite{Mecozzi_2018} it is shown that the spectra efficiency of a band limited system under DD is at most \SI{1}{bit/\s/\Hz} less than the same system under coherent detection. Also in \cite{Tasbihi_Capacity} it is proven that for time limited signals  the capacity is also at most one bit les than the coherent case. This means that contrary to intuition, the loss in the capacity of a system under DD is not a half of the coherent system, but just \SI{1}{bit/\s/\Hz}.\\

The results of this papers show that the systems with DD have a big potential, because the detector are cheaper and easier to implement (basically just one photodiode) and the loss in the capacity may not be as big as thought. However the problem to find a system simple enough that uses the potential of the DD is still open. 



\section{Trivial examples of phase retrieve from the magnitud}

% plabst BPSK secondini as image of tasbihi trivial example 

% tasbihi sinc example





























\chapter{Tukey signaling}
\chaptermark{Tukey signaling}


% system model

% numerical results 

% discussion 
	% Complexity 
	% bandwidth efficiency 
	% analog circuit may be difficult to implement 
\chapter{Generalized Direct Detection with phase recovery}
\chaptermark{Generalized  Direct Detection with phase recovery}
\newcommand{\PlabstImage}[1]{images/Tukey_Signaling/#1.pdf}

% system model (ct)
% system model (dt)
% SYMBOL-WISE MAP DETECTION
% numerical results 
% new constellation








\cleardoublepage
\phantomsection
\addcontentsline{toc}{chapter}{\bibname} 
\printbibliography



\end{document}

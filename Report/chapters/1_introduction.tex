\chapter{Introduction}
\chaptermark{Introduction}
\label{ch:introduction}
 % to change the path of the folder with the images of this chapter
\newcommand{\IntroImage}[1]{images/intro/#1}


	

Square-law (SQL) detection, also known as Direct detection (DD), is a detection scheme that measures the square magnitude of a complex waveform; clearly, this scheme is a nonlinear waveform detection because of the square operation. Direct detection is widely used in different scientific fields, such as crystallography, radio astronomy, and biomedical spectroscopy, among others \cite{Tasbihi_Tukey}. In particular, this scheme is often used in optical communications, especially in short-haul systems with length up to \SI{50}{\km} \cite{Agrawal_ch1} due to its simplicity, or even in shorter links up to \SI{10}{\km}, for example in rack to rack communications in big data centers \cite{Tasbihi_Tukey}.
\nomenclature[E21]{DD}{Direct Detection}{}{}%
\nomenclature[E22]{SQL}{Square law}{}{}%

In recent years, the interest in studying systems with DD is getting bigger again because of the simplicity of the receivers, which are a promising low-cost alternative compared to the coherent detection systems \cite{Mecozzi_2018}. In this context, two questions or problems arise. First, how good is a system with DD compared to a system with coherent detection in terms of the information capacity of the channel. And second, how to design a system that exploits the information capacity of the DD channel in the best possible way.

In this work, we briefly cover two answers to the first question, and then we review two systems proposed for a channel with DD. Then, we try to propose a new decoder for the second system, with reduced complexity at the expense of a slightly worse performance.


\section{Direct Detection}
\label{sec:Direct_Detection}

In optical communication, DD is performed with a single photodiode that converts the optical signal into an electric signal through the photoelectric effect according to the next equation \cite{Agrawal_ch4}:
\begin{equation}
I_p = R_d\cdot P_{in}
\label{eq:photocurrent}
\end{equation}
where $I_p$ is the photocurrent, $P_{in}$ is the incident optical power (which is proportional to the square of the magnitude of the electric field, that is where the square-law term comes from), and $R_d$ is the so-called responsivity of the photodetector, with units of \SI{}{\A/\W}.

The noise in the photodiode is generated primarily by two mechanisms: in the first place, the shot noise, and in the second place, thermal noise.

The shot noise models the fact that the photocurrent consists of a stream of electrons generated at random times. Mathematically, the current corresponding to the shot noise $i_s(t)$ is a stationary random process with Poisson distribution but is usually approximated by a Gaussian distribution with variance given by \cite{Agrawal_ch4}:
\begin{equation}
\sigma_s^2 = 2qI_pB
\label{eq:shot_noise_varaince}
\end{equation}%
\nomenclature[C18]{$q$}{Electron charge}{}{}%
where $q$ is the electron charge, $I_p$ the photocurrent, and $B$ the bandwidth of the system. $\sigma_s$ can be interpreted as the RMS value of the shot noise current $i_s(t)$.

The thermal noise is generated by the movement  of electrons due to the ambient temperature; this kind of noise is Gaussian distributed, and its variance  is given by \cite{Agrawal_ch4}:
\begin{equation}
\sigma_{th}^2 = \frac{4k_BT}{R_L} F_nB
\label{eq:thermal_noise_variance}
\end{equation}%
\nomenclature[C17]{$k_b$}{Boltzmann constant}{}{}%
where $k_B$ is the Boltzmann constant, $T$ is the temperature given in kelvin, $R_L$ is the load resistance, $B$ the bandwidth, and $F_n$ is the amplifier noise figure. 

With this in mind, the output current of the direct detection process is given by:
\begin{equation}
I(t) = I_p+i_s(t)+i_{th}(t)
\label{eq:DD_current}
\end{equation}
with $i_s(t)\sim\mathcal{N}(0,\sigma_s^2)$ and $i_{th}(t)\sim\mathcal{N}(0,\sigma_{th}^2)$.



\section{Capacity under direct detection}
\label{sec:capacity_under_direct_detection}

A communications channel that uses DD can retrieve only the information about the magnitude of the signal; in contrast, a system with coherent detection can retrieve the magnitude and phase of the signal. This means that the DD scheme ignores one of the two degrees of freedom, and hence, it is reasonable to think that the capacity of this system should be approximately half that of the system with coherent detection \cite{Mecozzi_2018, Tasbihi_Tukey, Tasbihi_Capacity}.

However, in \cite{Mecozzi_2018}, it is shown that the spectra efficiency of a band-limited system under DD is at most \SI{1}{bit/\s/\Hz} less than the same system under coherent detection. Also, in \cite{Tasbihi_Capacity}, it is proven that for time-limited signals, the capacity is also at most one bit less than the coherent case. This means that contrary to intuition, the loss in the capacity of a system under DD is not half of the coherent system, but just \SI{1}{bit/\s/\Hz}.

The results of these papers show that the systems with DD have a considerable potential because the detector is cheaper and easier to implement (basically just one photodiode), and the loss in the capacity may be smaller than thought. However, the problem of finding a system simple enough that uses the potential of the DD is still open. 



\section{Basic principle of phase recovery}

The key to retrieve the phase information of the transmitted symbols when using DD is to use the ISI. As a toy example, one can think of the problem where given two complex numbers $z_1$ and $z_2$; from $|z_1|^2$, $|z_2|^2$ and $|z_1+z_2|^2$ (an ISI term), it is possible to determine the phase difference between $z_1$ and $z_2$ up to a sign ambiguity \cite{Tasbihi_Tukey}.%
\nomenclature[E25]{ISI}{Inter symbol Interference}{}{}%

To show this, notice the following:
\begin{align*}
	|z_1+z_2|^2 &= (z_1+z_2)(z_1+z_2)^* \\
	&=z_1z_1^*+z_1z_2^*+z_2z_2^*+z_1^*z_2\\
	&=|z_1|^2+|z_2|^2+z_1z_2^*+\bigl(z_1z_2^*\bigr)^*\\
	&=|z_1|^2+|z_2|^2+2\text{Re}\{z_1z_2^*\}\\
\end{align*}%
\nomenclature[B06]{Re$\{z\}$}{Real part of $z$}{}{}%
\nomenclature[B07]{Im$\{z\}$}{imaginary part of $z$}{}{}%
under the convention that $z_1 = a\cdot e^{j\alpha}$ and $z_2 = b\cdot e^{j\beta}$%
\nomenclature[C16]{$j$}{Imaginary unit}{}{}%
\begin{equation}
	|z_1+z_2|^2 =|z_1|^2+|z_2|^2+2|z_1||z_2|\cos(\alpha-\beta)\\
	\label{eq:square_mag_of_sum}
\end{equation}

Clearly if $|z_1|^2$, $|z_2|^2$ and $|z_1+z_2|^2$ are known, one can solve the equation \ref{eq:square_mag_of_sum} for $\cos(\alpha-\beta)$ and get the information about the phase difference between $z_1$ and $z_2$. However, since cosine is an even function $\cos(\alpha-\beta)=\cos(-\alpha+\beta)$, hence there is still an ambiguity on the sign of the phase difference.


\begin{figure}[htb]
     \centering
     \begin{subfigure}[b]{0.49\textwidth}
         \centering
         \includegraphics[width=\textwidth]{\IntroImage{CD_toy_example.pdf}}
         \caption{Coherent detection}
         \label{fig:CD_toy_example}
     \end{subfigure}
     \hfill
     \begin{subfigure}[b]{0.49\textwidth}
         \centering
         \includegraphics[width=\textwidth]{\IntroImage{DD_toy_example.pdf}}
         \caption{Direct detection}
         \label{fig:DD_toy_example}
     \end{subfigure}
     \hfill
     \vspace{10mm}
     \begin{subfigure}[b]{\textwidth}
         \centering
         \includegraphics[width=0.8\textwidth]{\IntroImage{DD_toy_example_long_sq.pdf}}
         \caption{Direct detection of the sequence $\{+1,-1,-1,+1,+1,+1,-1\}$}
         \label{fig:DD_toy_example_long_sq}
     \end{subfigure}
        \caption{Comparative of the waveforms under coherent detection and direct detection. Based on \cite{Secondini, Plabst_DD}.}
        \label{fig:DD_vs_CD}
\end{figure}

To visualize this principle in a real system (jet not even near optimal due to the big bandwidth of the pulse), see figure \ref{fig:DD_vs_CD}. There is shown with a solid line the waveform of a simple transmission under coherent detection (see figure \ref{fig:CD_toy_example}) and under direct detection (see figure \ref{fig:DD_toy_example}), and also the waveform of the individual transmitted symbols in dashed lines.

For the coherent detection case, the samples at each symbol time are sufficient to determine the transmitted symbols. In contrast, for the direct detection case, the samples at each symbol time (circles) only carry information about the magnitude of the symbols (always one in this example), and the samples at intermediate symbol time (squares) carry information about the phase difference \cite{Secondini}.

Figure \ref{fig:DD_toy_example_long_sq} shows the DD system for a longer sequence. There it is possible to notice that the even samples always carry information about the magnitude, whereas the odd samples carry information about the differential phase of the symbols. This toy example shows the principle behind phase recovery in DD systems.

It is important to highlight two things: first, an oversampling factor of two is needed, and second, only information on the differential phase (up to a sign ambiguity) is retrieved. Hence, it is beneficial to use differential coding in the transmission.

Another way to explain the previous conclusions is to take a look at the simple system given by:
\begin{align*}
	g(t)=\sum_{k=0}^m g_k\text{sinc}(t-k)
\end{align*}
where $g_k$ is a complex number representing the $k$-th transmitted symbol, and $g(t)$ is the transmitted signal. Note that after the DD, the bandwidth of the signal is twice as big as the bandwidth of $g(t)$ since the received signal is given by the product $|g(t)|^2=g(t)g^*(t)$. Hence, to recover the data, one should use an oversampling factor of two, so the sampled signal becomes \cite{Tasbihi_Tukey}:
\begin{align*}
	\left|g\left(\frac{n}{2}\right)\right|^2 = \left\{
\begin{array}{ll}
\left|g_{\frac{n}{2}}\right|^2  &  \text{if $n$ is even}  \\
\left|\sum\limits_{k=0}^m g_k\text{sinc}\left(\frac{n}{2}-k\right)\right|^2   & \text{if $n$ is odd}
\end{array}
\right. &&\text{for }n=0,\dotsb,2m
\end{align*}%
\nomenclature[B12]{sinc$(t)$}{$\frac{\sin(\pi t)}{\pi t}$}{}{}%

Once again, it is clear that the even samples carry the information about the magnitude of the symbols. In contrast, the odd samples have somehow the information about the phase of the symbols, but now notice that all the $g_k$ contribute to the odd samples. As we will see in later chapters, retrieving the phase quickly becomes an intractable problem as $m$ grows \cite{Tasbihi_Tukey}. 


















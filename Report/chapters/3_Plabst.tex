\chapter{Generalized Direct Detection with phase recovery}
\chaptermark{Generalized  Direct Detection with phase recovery}
\label{ch:Plabst}
\newcommand{\PlabstImage}[1]{images/Plabst/#1}


The other solution considered for the phase recovery problem is the one proposed by Plabst in the paper \cite{Plabst_DD}. The authors try to improve the two drawbacks of the Tukey signaling scheme. First, they considered an arbitrary pulse waveform, and in particular, they experimented with raised cosine waveforms to improve the spectral efficiency. Second, they propose a discrete-time model, which allows the use of a digital system with an oversampling factor of only two.

The proposed system model can be seen in figure \ref{fig:Plabst_system_model}. In the following sections, we will explain the system model, both in continuous and discrete time, and the detector used to determine an upper bound limit of the system's performance.

\begin{figure}[htbp]
\begin{center}
\includegraphics[width=\textwidth]{\PlabstImage{Plabst_system_model.pdf}}
\caption{System model proposed in \cite{Plabst_DD}.}
\label{fig:Plabst_system_model}
\end{center}
\end{figure}
\nomenclature[E23]{SLD}{Square law detection}{}{}

\section{System model}

\subsection{Continuous time Model}

\subsubsection{Differential encoder}

As it was mentioned in chapter \ref{ch:Tukey_signaling} the DD generates some ambiguities, and one way to eliminate them is by using differential encoding \cite{Plabst_DD,Plabst_SIC,Secondini}. As suggested in \cite{Plabst_SIC}, the phase encoding is done by a function that receives as an input a vector of symbols $\bm u$ and outputs a vector of symbols $\bm x=f_\text{diff}(\bm u)$ with the same magnitude as the symbols in $\bm u$, and a given phase based on a set of conditions, the phase of the input symbol and the phase of the last output symbol, as shown for example in figure \ref{fig:diff_encoding} for a M-ASK constellation.

\begin{figure}[htbp]
\begin{center}
\includegraphics[width=0.5\textwidth]{\PlabstImage{diff_map_M_ASK.pdf}}
\caption{Differential phase mapper example for M-ASK. Based on \cite{Plabst_SIC}.}
\label{fig:diff_encoding}
\end{center}
\end{figure}



\subsubsection{Transmitter}

The transmitter receives as input a sequence $\bm x=\{x_0,\dotsc,x_{n-1}\}$ of independent and identically distributed (i.i.d.) symbols $x_i$ taken from a finite constellation $\mathcal K=\{a_1,\dotsc,a_q\}$ with $q$ elements, and outputs a signal given by \cite{Plabst_DD}:%
\nomenclature[A01]{i.i.d.}{independent and identically distributed}{}{}%
\begin{equation}
x(t)=\sum_{k=0}^{n-1}x_k\cdot g_\text{tx}(t-iT)
\label{eq:Plabst_signaling_block}
\end{equation}
where $T$ is the inverse of the symbol rate, $x_i\in\mathcal K$, and $g_\text{tx}(t)$ is the pulse waveform.


Common pulse waveforms are raised cosine pulses. Of particular interest is the case for roll-off factor $\alpha=0$, that is:
\begin{align}
	g_\text{tx}&=\frac{t}{T_s}\text{sinc}\left(\frac{1}{T_s}\right)
	\label{eq:sinc_pulse_TD}\\
	G_\text{tx}&=\left\{
\begin{array}{ll}
1  &  \text{if }|f|\leq\frac{1}{2T_s}   \\
0  &  \text{otherwise} \\
\end{array}
\right.
\label{eq:sinc_pulse_FD}
\end{align}

\subsubsection{Fiber Optic Link}

The channel considered is an optic fiber that only presents chromatic dispersion, characterized in the frequency domain by \cite{Plabst_DD}:
\begin{equation}
H_L(f)=e^{j2\beta_2L\pi^2f^2}
\end{equation}
where $\beta_2$ is the group-velocity dispersion parameter and $L$ is the fiber length.

\subsubsection{Receiver}
The receiver consists of a photodiode, whose output is given by:
\begin{equation}
z(t)=|x_L(t)|^2
\end{equation}

The photodiode noise is modeled as a real-valued white Gaussian process with spectral density $N_0/2$ \cite{Plabst_DD}, representing the thermal noise. Finally, the receiver has a band-limited sampler, with an impulse response given by: 
\begin{align}
	g_\text{rx}&=2B\text{sinc}\left(2Bt\right)\\
	G_\text{rx}&=\left\{
\begin{array}{ll}
1  &  \text{if }|f|\leq B   \\
0  &  \text{otherwise} \\
\end{array}
\right.
\end{align}
where $B$ is the symbol rate given by $B=1/T_s$ \cite{Plabst_DD}.



\subsection{Discrete time Model}

\subsubsection{Discrete signal notation}
For the discrete-time model, a receiver with sampling rate $1/T'_s=2B$ is considered (in this section, the super index $'$ denotes a variable of the oversampled system), that is an oversampling factor of $N_\text{os}=T_s/T'_s=2$, meaning two samples per transmitted symbol \cite{Plabst_DD}.

Let $\bm x'=\{0,x_0,\dotsc0,x_{n-1}\}$ and $\bm y'=\{y_0,\dotsc,y_{2n-1}\}$ be the upsampled sequence at the transmitter and receiver, so they are related by \cite{Plabst_DD}:
\begin{align}
y'_k&=z'_k+n'_k \\
z'_k&=\left(|x_L(t)|^2*g_\text{rx}(t)\right)_{t=kT'_s}
\end{align}
where $n_k\sim\mathcal N(0,\sigma_N^2)$ and $\sigma_N^2=N_0B$.

Now let us define the combined impulse response of the pulse shaping filter and the channel response as $\uppsi(t)=g_\text{tx}(t)*h_L(t)$ and the discrete version of it $\uppsi_k=\uppsi(kT'_s)$, then \cite{Plabst_DD}:
\begin{equation}
x_L(kT'_s)=\sum_{m=0}^{2n-1}\uppsi_mx'_{k-m}
\end{equation}

Finally, notice that, for the special case when the pulse shape is the sinc pulse in equations \ref{eq:sinc_pulse_TD} and \ref{eq:sinc_pulse_FD}, $|x_L(t)|^2*g_\text{rx}=|x_L(t)|^2$, so:
\begin{equation}
z'_k=\left|\sum_{m=0}^{2n-1}\uppsi_mx'_{k-m}\right|^2
\end{equation}


\subsubsection{Vector matrix notation}

Now, if we think of the signals as column vectors:
\begin{align*}
	\bm x'&=\bigl[0,x_0,\dotsc,0,x_{n-1}\bigr]^T &&\in\mathds C^{2n\times1}\\
	\bm z'&=\bigl[z_0,\dotsc,z_{2n-1}\bigr]^T &&\in\mathds R^{2n\times1}\\
	\bm n'&=\bigl[n_0,\dotsc,n_{2n-1}\bigr]^T &&\in\mathds R^{2n\times1}\\
	\bm y'&=\bigl[y_0,\dotsc,y_{2n-1}\bigr]^T &&\in\mathds R^{2n\times1}
\end{align*}

And consider that $\uppsi$ is time limited, so $\uppsi_m$ is zero outside some interval $[0,M-1]$, let us define the Toeplitz matrix
$\bm \Uppsi \in\mathds C^{2n\times(2n+M-1)}$ as:
\begin{equation}
	\bm \Uppsi = \begin{bmatrix}
				\uppsi_{M-1}&\uppsi_{M-2}&\cdots &\uppsi_0&&&\\
				&\uppsi_{M-1}&\cdots &\uppsi_1&\uppsi_0&&\\
				&&\ddots&&&\ddots&&\mbox{\Huge 0}&\\
				&&&\uppsi_{M-1}&\cdots&\cdots&\uppsi_0&\\
				&\mbox{\Huge 0}&&&\ddots&&&\ddots&&\\
				&&&&&\uppsi_{M-1}&\cdots&\cdots&\uppsi_0&\\
			     \end{bmatrix}
\end{equation}

Finally define the channel state $\bm s_0$ as:
\begin{equation}
\bm{s}'_0=[0,x_{-\widetilde{M}},0,x_{1-\widetilde{M}},\dotsc,0,x_{-1}]^T \in \mathds{C}^{(M-1)\times1}
\end{equation}
where $\widetilde{M}$ is the memory channel in terms of the transmitted symbols and is given by $\widetilde{M}=(M-1)/2$.


With the previous definitions, the output of the square law detection is given by:
\begin{align*}
	\bm z' = \left|\bm \Uppsi \left[
\begin{array}{c}
\bm{s}'_0  \\
   \bm x'
\end{array}
\right]
\right|^{\circ2} = \left|\bm{ \Uppsi \tilde{x'}}\right|^{\circ2} \qquad \in\mathds{R}^{2n}
\end{align*}
where $|\cdot|^{\circ2}$ is the element wise $|\cdot|^2$ operator.
\nomenclature[B15]{$\lvert\cdot\rvert^{\circ2}$}{Element wise $\lvert\cdot\rvert^2$ operator}{}{}

Finally, the complete discrete-time system model, including the Gaussian noise, is given by \cite{Plabst_DD}:
\begin{align*}
	\bm y' = \bm z' + \bm n' = \left|\bm{ \Uppsi \tilde{x'}}\right|^{\circ2} +\bm n' \qquad \in \mathds{R}^{2n}
\end{align*}
and the channel’s conditional probability density is Gaussian\cite{Plabst_DD}:
\begin{equation}
p(\bm y'|\bm x')=\mathcal N \left( \bm y- \left|\bm{ \Uppsi \tilde{x'}}\right|^{\circ2}; \bm 0_{2n} , \sigma_N^2 \bm I_{2n}  \right)
\end{equation}
\nomenclature[A04]{$\mathcal N(x;\mu,\sigma^2)$}{Univariate real Gaussian distributions, with mean $\mu$ and variance $\sigma^2$.}{}{}
\nomenclature[A05]{$\mathcal N(\bm x;\bm\mu,\bm C)$}{Multivariate real Gaussian distributions, with mean vector $\mu$ and covariance matrix $\bm C$.}{}{}
\nomenclature[B08]{$\bm 0_{n}$}{Ceros vector of length $n$}{}{}
\nomenclature[B09]{$\bm I_{n}$}{Identity matrix of size $n\times n$}{}{}


\subsubsection{Even and odd samples}

For convenience one can concatenate the $k$-th and $(k+1)$-th sample in a new vector as follows:
\begin{align}
\bm z_k&= \bigl[z'_{2k},z'_{2k+1}\bigr]=\bigl[z_{k}^{\text{e}},z_{k}^{\text{o}}\bigr]
\label{eq:z_even_odd}\\
\bm y_k&=\bigl[y_{k}^{\text{e}},y_{k}^{\text{o}}\bigr]=\bm z_k+\bigl[n_{k}^{\text{e}}, n_{k}^{\text{o}}\bigr]
\label{eq:y_even_odd}
\end{align}

The reason for group two samples can be understood by looking at the figure \ref{fig:Exp_even_odd_samp}. The convolution and the SQL detection are represented for an even and odd sample in the case of link length $L=0$. For this choice of waveform, the even sample is an ISI-free sample and carries information about the magnitude of a past symbol; in contrast, the odd sample experiences ISI and, hence, somehow has information about the phases of the symbols.  


\begin{figure}[htbp]
\begin{center}
\includegraphics[width=0.9\textwidth]{\PlabstImage{Exp_even_odd_samp.pdf}}
\caption{Signals for oversampling factor $N_\text{os} = 2$. Based on \cite{Plabst_DD}.}
\label{fig:Exp_even_odd_samp}
\end{center}
\end{figure}

\subsection{Symbol-wise MAP detection}
\nomenclature[E28]{MAP}{Maximum a posteriori probability}{}{}
\nomenclature[E29]{APP}{a posteriori probability}{}{}
To evaluate the optimal performance of the system, in \cite{Plabst_DD} is proposed a decoder based on the maximum a posteriori rule that decides the transmitted symbol based on the a posteriori probabilities (APP) $p(x_k|\bm y')$. Since the APPs are proportional to $p(x_k,\bm y')$ because $p(\bm y')$ is constant with respect to the decision, the MAP rule becomes \cite{Plabst_DD}:
\begin{equation}
\hat{x}_k=\argmax_{x_k}p(x_k,\bm y')
\label{eq:MAP_deco}
\end{equation}
as a reminder, $\bm y'$  is the hole vector of received  samples, i.e. $\bm y'=\bigl[y_0,\dotsc,y_{2n-1}\bigr]^T$, and $p(x_k,\bm y')$ is given by \cite{Plabst_DD}:
\begin{equation}
p(x_k,\bm y')=\sum_{s_0}\sum_{\bm x\backslash x_k} p(s_0,\bm x, \bm y')
\label{eq:APPs}
\end{equation}
and $\sum_{\bm x\backslash x_k}$ denotes a sum over all possible vectors $\bm x$  but where the $k$-th position is fixed.

These APPs can be efficiently computed with the Forward-Backward Algorithm or BCJR algorithm with the factor graph shown in figure \ref{fig:BCJR} and the following recursive equations \cite{Plabst_DD}:

\begin{figure}[htbp]
\begin{center}
\includegraphics[width=\textwidth]{\PlabstImage{BCJR.pdf}}
\caption{Factor graph to implement the BCJR. Based on \cite{Plabst_DD}.}
\label{fig:BCJR}
\end{center}
\end{figure}



\begin{align}
	\vv{\mu}(\bm s_k)&=\sum_{x_k,\bm s_{k-1}}p(x_k,\bm y_k,\bm s_k|\bm s_{k-1})\cdot \vv{\mu}(\bm s_{k-1})&&\text{FW path}
	\label{eq:FW}\\
	\cev{\mu}(\bm s_{k-1})&=\sum_{x_k,\bm s_{k}}p(x_k,\bm y_k,\bm s_k|\bm s_{k-1})\cdot \cev{\mu}(\bm s_{k})&&\text{BW path}
	\label{eq:BW}\\
	p(x_k,\bm y')&=\sum_{\bm s_{k-1}}\sum_{\bm s_k}\vv{\mu}(\bm s_{k-1})\cdot p(x_k,\bm y_k,\bm s_k|\bm s_{k-1})\cdot\cev{\mu}(\bm s_{k})&&\text{APP}
	\label{eq:APP}
\end{align}
where $k=1,\dotsc,n$, $\vv{\mu}(\bm s_0)=p(\bm s_0)$ and $\cev{\mu}(\bm s_n)=p(\bm s_n)$.

Finally, the likelihood is given by \cite{Plabst_DD}:
\begin{equation}
p(x_k,\bm y_k,s_k|\bm s_{k-1})=\underbrace{p(\bm y_k|x_k,\bm s_{k-1})}_{(a)}\;
\underbrace{p(\bm s_k|x_k,\bm s_{k-1})}_{(b)}\;
\underbrace{p(x_k|\bm s_{k-1})}_{(c)}
\end{equation}
where $(a)$ is:
\begin{align}
	p(\bm y_k|x_k,\bm s_{k-1})&=p(y_k^\text{e}|\bm s_{k-1})\cdot p(y_k^\text{o}|x_k,\bm s_{k-1})\\
	p(y_k^\text{e}|\bm s_{k-1})&=p_N\left(y_k^\text{e}-\left| [0,x_{k-\widetilde{M}},\cdots,0,x_{k-1},0]\cdot\uppsi\right|^2\right)
	\label{eq:likelihood_even}\\
	p(y_k^\text{o}|x_k,\bm s_{k-1})&=p_N\left(y_k^\text{o}-\left| [x_{k-\widetilde{M}},0,\cdots,x_{k-1},0,x_{k}]\cdot\uppsi\right|^2\right)
	\label{eq:likelihood_odd}
\end{align}
The term $(b)$ is 1 if the state transition from $\bm s_{k-1}$ to $\bm s_{k}$ is possible and 0 otherwise, and $(c)$ is $p(x_k|\bm s_{k-1})=p(x_k)$ since the symbols are i.i.d. \cite{Plabst_DD}.

Additionally, since using differential encoding is recommended to avoid ambiguities, one can include the differential decoding in the BCJR algorithm by including the fact that $\bm x=f_\text{diff}(\bm u)$ in the factor graph \cite{Plabst_DD}. That is done by doing the summations in equations \ref{eq:FW}, \ref{eq:BW} and \ref{eq:APP} overall $u_k$ instead of $x_k$ (which in terms of the algorithm makes no change), and defining the states of equations \ref{eq:likelihood_even} and \ref{eq:likelihood_odd} in terms of the $\bm u$, that means replacing $\bm x$ by $f_\text{diff}(\bm u)$ (for more details see the algorithm used in \cite{Wang}).  


\section{Numerical simulation}
\nomenclature[E31]{SER}{Symbol error rate}{}{}
For the numerical simulations we implemented the proposed system model in a Python simulation, which can be found in the git repository \href{https://github.com/dfigueroa11/Direct_Detection_with_Phase_Recovery.git}{Direct\_Detection\_with\_Phase\_Recovery}. As the system parameters we use the settings found in table \ref{tab:sim_Plabst}, and we estimate the SER using a Montecarlo simulation with \SI{20000} transmitted symbols. We set the noise power to $\sigma_N^2=1$ so the SNR is given by \cite{Plabst_DD}:
\begin{equation}
\text{SNR}=P_\text{tx}=\frac{\mathds{E}\bigl[ ||x(t)||\bigr]}{nT_s}
\end{equation}
\nomenclature[A02]{$\mathds E [X]$}{Expected value of $X$}{}{}%

\begin{table}[htp]
\begin{center}
\begin{tabular}{|c|c|}\hline
Paramater&Value\\\hline\hline
$\beta_2$&\SI{-2.168e-23}{s^2/km}\\\hline
attenuation factor&\SI{0.2}{dB/km}\\\hline
link lengths $L$&\SI{0}{km}\\\hline
Baud rate $B$& \SI{35}{Gbaud}\\\hline
\end{tabular}
\end{center}
\caption{Simulation parameters. Taken from \cite{Plabst_DD}.}
\label{tab:sim_Plabst}
\end{table}%



For the pulse shaping filter, we used a sinc pulse with a sampling rate of $2B$. Since the complexity of the decoder grows exponentially with the length of the filter, we could not use in the decoder the complete signal $\uppsi(t)$, instead of that we use an auxiliary channel for the decoding process that considers only a window of the real channel as shown in figure \ref{fig:aux_channel}. That means, we simulated the transmission with a channel with memory (in terms of the transmitted symbols) of $\widetilde{M}$, i.e. $M=2\widetilde{M}+1$ taps, but decode with an auxiliary channel with memory $\widetilde{N}$ i.e. $N=2\widetilde{N}+1$ taps. 


\begin{figure}[htbp]
\begin{center}
\includegraphics[width=0.6\textwidth]{\PlabstImage{aux_channel.pdf}}
\caption{Simulated and auxiliary channel. Based on \cite{Plabst_DD}.}
\label{fig:aux_channel}
\end{center}
\end{figure}

For the constellations, we choose three different constellations, BPSK, 4-QAM, and DD-SQAM, a specially designed constellation,  shown in figure \ref{fig:DD_SQAM_constellation}, to mitigate the ambiguities that arise from the DD. As pointed out in section \ref{sec:ambiguities}, constellations with high symmetry produce a lot of ambiguities, and the DD process only distinguishes the phase difference between symbols through the cosine of the phase difference, so when designing the DD-SQAM constellation, we reduced the symmetry, and separate the cosine of the angle between symbols as much as possible, that is why we use the particular angle $\phi=\arccos(1/3)$.

\begin{figure}[!ht]
\begin{center}
\includegraphics[width=0.38\textwidth]{\PlabstImage{DD_SQAM_constellation.pdf}}
\caption{DD-SQAM constellation.}
\label{fig:DD_SQAM_constellation}
\end{center}
\end{figure}

Finally, we simulate two scenarios for each constellation, one when the auxiliary channel is much shorter than the actual channel and the other when the auxiliary and the actual channels have the same length. The length of the auxiliary channel is chosen based on complexity criteria; we choose a relatively large length that is still computable in a reasonable time. The results can be seen in figure \ref{fig:Plabst_result_SER}.

\begin{figure}[!ht]
\begin{center}
\includegraphics[width=0.8\textwidth]{\PlabstImage{Plabst_result_SER.pdf}}
\caption{Symbol error rate vs SNR for different constellations and channel lengths.}
\label{fig:Plabst_result_SER}
\end{center}
\end{figure}

Looking at the figure \ref{fig:Plabst_result_SER}, we can notice that the simulations with a channel miss-match ($\widetilde{N}<\widetilde{M}$) have an error floor, this is due to the ISI that is not considered by the BCJR decoder, so even increasing the SNR, the SER does not improve, because the main source of ``noise'' is the ISI.

It is also important to notice that the QAM constellation with $\widetilde{N}=\widetilde{M}$ also has an error floor at approximately 0.25; this is caused by the ambiguities generated by the DD for the case of the QAM constellation. Even with differential encoding, there are still ambiguities $1/4$ of the time because a phase change of $\pi$ or $0$ can be distinguished, but a phase change of $\pm\pi/2$ can not. Finally, it is important to notice that the BPSK and DD-SQAM constellations do not have an error floor when $\widetilde{N}=\widetilde{M}$. This shows that these constellations do not generate ambiguities when used with differential encoding.


\section{Discussion}

The system proposed in \cite{Plabst_DD} compared with the system proposed in \cite{Tasbihi_Tukey} has the advantage of using a waveform with a smaller bandwidth that increases the spectral efficiency \cite{Plabst_DD}, and also represents a simpler system compatible with a digital system with an oversampling factor of only 2, which is very desirable. However, the problem of finding a good decoder with feasible complexity is still open. 

For this reason, we use this system as a base to formulate a machine learning-based decoder, which will be discussed in the next chapter.













